\documentclass[12pt]{article}

\usepackage[utf8]{inputenc}
\usepackage[polish]{babel}
\usepackage[T1]{fontenc}
\usepackage{amsmath}
\usepackage{amsfonts}
\usepackage{anysize}
\usepackage{fancyhdr}
\usepackage{setspace}
\usepackage{url}
\usepackage{graphicx}
\usepackage{minibox}
\usepackage{paralist}
\usepackage{framed}
\usepackage{listings}
\usepackage{color}
\usepackage{float}
\usepackage{array}
\usepackage{tgpagella}
\usepackage[scaled]{helvet}
\usepackage{titlesec}
\usepackage{graphicx}
\usepackage{hyperref}

\renewcommand\thesection{\arabic{section}.}
\renewcommand\thesubsection{\arabic{section}.\arabic{subsection}.}
\renewcommand\thesubsubsection{\arabic{section}.\arabic{subsection}.\arabic{subsubsection}.}
\renewcommand{\footrulewidth}{0.4pt}% Default \footrulewidth is 0pt

\marginsize{2cm}{2cm}{2cm}{2cm}
\setlength{\parindent}{0pt}


\title{\textbf{Zarządzanie projektami informatycznymi \\ Semestr 15Z}}
\author{Katarzyna Olszewska\\
		Wiktor Ślęczka\\
		Łukasz Korpal}
\date{}
\begin{document}
\maketitle

\newpage
\large{\textbf{Projekt systemu zarządzania systemem dronów dostawczych\\
Dokumentacja Projektowa}}

\section{Założenia projektu}
Projekt ma na celu stworzenie systemu nadzorującego i zarządzającego siecią baz i dron dostawczych dla firmy ProDelivery.
System ma za zadanie ustalanie tras dla dostawców w sposób najbardziej optymalny, biorąc pod uwagę szybkość i koszt transportu.
W związku z tym, że serwis będzie musiał działać niezawodnie i wydajnie oraz uwzględniać warunki pogodowe oraz czynniki losowe panujące na trasach dronów, zastosowane technologie muszą spełniać rygorystyczne kryteria.
Dodatkowym atutem przy ich wyborze stanowić będzie szybkość działania, jednak największą wagę należy przyłożyć do niezawodności i bezpieczeństwa osób mających kontakt z dronami oraz transportowanego ładunku.
System ma mieć na uwadzę wygodę użytkowników wysyłających i odbierających przesyłki - będzie to kluczowym warunkiem powodzenia systemu na rynku.


\section{Rozpoczęcie projektu}
Rozpoczynając projekt, należy zwrócić szczególną uwagę na stronę formalną, potrzebne dokumenty i kontakt z władzami administracyjnymi obszarów, na których system będzie stosowany.
Największą uwagę przy realizacji projektu będzie należało poświęcić bezpieczeństwu użytkowników i towarów, co powinno być brane pod uwagę na każdym etapie realizacji projektu.
\subsection{Aspekty}
\subsubsection{Aspekty biznesowe}
\begin{itemize}
\item Zmniejszenie kosztów przeznaczanych na zasoby ludzkie
\item Usprawnienie systemu dostaw
\item Zmniejszenie czasu potrzebnego na dostarczenie przesyłki
\item Uniezależnienie dostaw od warunków drogowych
\item Zbudowanie wizerunku innowacyjnej firmy
\end{itemize}

\subsubsection{Aspekty użytkowe}
\begin{itemize}
\item Krótszy czas oczekiwania na przesyłki
\item Dostawy o każdej porze dnia i nocy
\item Brak chamskich Kurierów/Kurierek
\item Brak kontaktu z kadrą pracowniczą
\end{itemize}

\subsubsection{Aspekty techniczne}
\begin{itemize}
\item System opraty o strukturę P2P
\item Duża ilość punktów odbioru/nadania przesyłek
\item Centralny system zarządzania trasami dronów
\item Oprogramowanie dronów
\item Stacje ładowania dronów i punkty serwisowe
\item Prototyp drona
\end{itemize}

\subsection{Wymagania}
\subsubsection{Wymagania funkcjonalne}
\begin{itemize}
\item Nadawanie przesyłki
\item Sprawdzanie stanu przesyłki
\item Odbieranie przesyłki z automatu
\end{itemize}
\subsubsection{Wymagania niefunkcjonalne}
\begin{itemize}
\item Przypisywanie przesyłki do drona
\item Bezpieczny transport przesyłki
\item Płynna obsługa systemu
\item Monitorowanie stanu i lokalizacji dron
\item Przetwarzanie przeyłki
\item Tworzenie optymalnych tras biorąc pod uwagę warunki pogodowe
\item Pobranie przesyłki
\item Zostawienie przesyłki w skrytce
\item Nawigacja dronami
\item Rejestracja nowych dronów, skrytek i punktów ładowania
\end{itemize}

\newpage
\section{Organizacja projektu}
\subsection{Komitet sterujący}
Nadzór nad powstającym projektem, a także podejmowanie decyzji o znaczeniu strategicznym leży w zakresie obowiązków komitetu sterującego.\\
W jego skład wchodzić będą:\\

\begin{tabular}{|p{4.5cm}|p{4.5cm}|p{4.5cm}|} \hline
Stanowisko & Pracownik & Obowiązki \\
\hline \hline
Przewodniczący komitetu sterującego & Piotr Zieliński, Prezes EasyProgramming & 
\begin{itemize}
\item Realizacja celów strategicznych
\item Zapobieganie i rozwiązywanie sytuacji kryzysowych
\end{itemize}\\
\hline
Sponsor projektu & Grzegorz Brzęczyszczykiewicz, członek zarządku ProDelivery &
\begin{itemize}
\item Przydzielanie środków finansowych
\item Wspieranie tworzenia systemu
\end{itemize}\\
\hline
Kierownik projektu & & 
\begin{itemize}
\item Tworzenie i nadzór harmonogramu prac
\item Wspieranie komunikacji i określanie jej zasad 
\item Pośrednictwo między kierownikami zespołów a komitetem sterującym
\item Zarządzanie ryzykiem
\end{itemize}\\
 \hline
\end{tabular}
\newpage
\begin{tabular}{|p{4.5cm}|p{4.5cm}|p{4.5cm}|} \hline
Główny użytkownik & & 
\begin{itemize}
\item Analiza zapotrzebowania przyszłych użytkowników
\item Doradztwo w zakresie użytkowania systemu
\end{itemize}\\
 \hline
Konsultanci techniczni & N/A & \\
 \hline
\end{tabular}

\end{document}
